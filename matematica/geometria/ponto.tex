A implementação do ponto como struct ajuda a otimizar algumas operações - por conta da sobrecarga dos operadores, mas perde-se mais tempo escrevendo. Sempre que possível, dar preferência para $par(double)$ e adaptar as funções de ponto para tal tipo - o que é bastante rápido. Operações com Vetores utilizam as funções de ponto.

\begin{lstlisting}[language=C++, title=Parte 1: Estrutura]
struct point {
    double x, y; point(): x(0), y(0) {}
    point(double _x, double _y): x(_x), y(_y) {}
    
    double norm() { return hypot(x, y); }
    
    point normalized() {
        return point(x, y) * (1.0 / norm()); }
        
    double polarAngle() {
		double a = atan2(y, x);
		return a < 0 ? a + 2 * acos(-1.0) : a; }
		
	bool operator<(point other) const {
		if (fabs(x - other.x) > EPS) return x < other.x;
		else return y < other.y; }
    
	bool operator==(point other) const {
		return (fabs(x - other.x) < EPS && 
		       (fabs(y - other.y) < EPS)); }
    
	point operator+(point other) const {
		return point(x + other.x, y + other.y); }
    
	point operator-(point other) const {
		return point(x - other.x, y - other.y); }
		
	point operator*(double k) const {
		return point(x * k, y * k); }
};
\end{lstlisting}

\newpage

\begin{lstlisting}[language=C++, title=Parte 2: Operações]
double distance(point p1, point p2) {
	return hypot(p1.x - p2.x, p1.y - p2.y); }

double innerProduct(point p1, point p2) {
	return p1.x * p2.x + p1.y * p2.y; }

double crossProduct(point p1, point p2) {
	return p1.x * p2.y - p1.y * p2.x; }

bool ccw(point p, point q, point r) {
	return crossProduct(q - p, r - p) > 0; }

bool collinear(point p, point q, point r) {
	return fabs(crossProduct(p - q, r - p)) < EPS; }

point rotate(point p, double rad) {
	return point(p.x * cos(rad) - p.y * sin(rad), 
	             p.x * sin(rad) + p.y * cos(rad)); }

double angle(point a, point o, point b) {
	return acos(innerProduct(a - o, b - o) / 
	           (distance(o, a) * distance(o, b))); }

point proj(point u, point v) {
	return v * (innerProduct(u, v) / innerProduct(v, v)); }

bool between(point p, point q, point r) {
	return collinear(p, q, r) && inner(p - q, r - q) <= 0; }

point lineIntersectSeg(point p, point q, point A, point B) {
	double c = crossProduct(A - B, p - q);
	double a = crossProduct(A, B), b = crossProduct(p, q);
	return ((p - q) * (a / c)) - ((A - B) * (b / c)); }

bool parallel(point a, point b) {
	return fabs(crossProduct(a, b)) < EPS; }

\end{lstlisting}

\newpage

\begin{lstlisting}[language=C++, title=Parte 2: Operações (cont.)]
bool segIntersects(point a, point b, point p, point q) {
	if (parallel(a - b, p - q)) {
		return between(a, p, b) || between(a, q, b) || 
               between(p, a, q) || between(p, b, q); 
	}
	point i = lineIntersectSeg(a, b, p, q);
	return between(a, i, b) && between(p, i, q);
}

point closestToLineSegment(point p, point a, point b) {
	double u = inner(p - a, b - a) / inner(b - a, b - a);
	if (u < 0.0) return a; if (u > 1.0) return b;
	return a + ((b - a) * u); }

// Outra forma de saber o ccw
int ccw2(ponto &P0, ponto &P1, ponto &P2) {
    int dx1 = P1.x - P0.x, dx2 = P2.x - P0.x;
    int dy1 = P1.y - P0.y, dy2 = P1.y - P0.y;
    if (dy1 * dx2 > dy2 * dx1) return -1;
    if (dx1 * dy2 > dy1 * dx2) return 1;
    if ((dx1 * dx2 < 0) || (dy1 * dy2 < 0)) return 1;
    if ((dx1 * dx1 + dy1 * dy1) < (dx2 * dx2 + dy2 * dy2))
        return -1;
    return 0; }

// corssProduct p/ 3 pontos
int crossProduct2(ponto &O, ponto &A, ponto &B) {
	return (A.x - O.x) * (B.y - O.y) - 
	       (A.y - O.y) * (B.x - O.x); }

// usado p/ convexHull
bool angleCmp(point a, point b) {
	if (collinear(pivot, a, b))
		return inner(pivot - a, pivot - a) <
		       inner(pivot - b, pivot - b);
	return cross(a - pivot, b - pivot) >= 0;
}
\end{lstlisting}