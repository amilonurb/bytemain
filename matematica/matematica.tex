\subsubsection*{Propriedades do Módulo}
\begin{itemize}
	\item $\left(A + B\right) \;\%\; C = 
	      \left(\left(A \;\%\; C\right) + 
	      \left(B \;\%\; C\right)\right) \;\%\; C$
	\item $\left(A - B\right) \;\%\; C = 
	      \left(\left[\left(A \;\%\; C\right) -
	      \left(B \;\%\; C\right)\right] + C\right) \;\%\; C$
	\item $\left(A \times B\right) \;\%\; C = 
	      \left(\left(A \;\%\; C\right) \times
	      \left(B \;\%\; C\right)\right) \;\%\; C$
	\item $\left(A \div B\right) \;\%\; C = 
	      \left(\left(A \;\%\; C\right) \times
	      \left(B^{-1} \;\%\; C\right)\right) \;\%\; C$
	\item $A^{B} \;\%\; C = \left(A \;\%\; C\right)^{B} \;\%\; C$
\end{itemize}

\subsection{Exponenciação Rápida}
\begin{lstlisting}[language=C++]
big int fastPower(bigint x, bigint n, bigint MOD) {
    big int y = x;
    big int result = 1;
    while (n > 0) {
        if (n & 1) result = (result % MOD * y % MOD) % MOD;
        y = (y % MOD * y % MOD) % MOD;
        n >>= 1;
    }
    return result;
}
\end{lstlisting}

\subsection{Números Primos}
\begin{lstlisting}[language = C++, title=Crivo de Eratóstenes]
V(int) sieve; V(int) primes; V(bool) is_primes;
void generatePrimes(int n) {
    V(bool) prime(n + 1, T);
    prime[0] = prime[1] = F;
    for (int i = 2; i * i <= n; i++) {
        if (prime[i]) {
            for (int j = i * i; j <= n; j += i) prime[j] = F;
            primes.PUB(i);
        }
    }
    is_primes = prime;
}
\end{lstlisting}

\newpage

\begin{lstlisting}[language = C++,title=Teste de Primalidade (Determinístico)]
bool isPrime(int number) {// V.1
    if (number <= sieve.size()) return is_primes[number];
    for (int i = 0; i < primos.size(); i++)
        if (number % primos[i] == 0) return F;
    return T;
}
bool isPrime(big n) {// V.2
	if (n <= 1) return F; if (n <= 3) return T;
    if ((n % 2 == 0) || (n % 3 == 0)) return F;
    for (int i = 5; (i * i) <= n; i += 6)
        if ((n % i == 0) || (n % (i + 2) == 0)) return F;
    return T;
}
\end{lstlisting}

\newpage

\begin{lstlisting}[language=C++, title={Fatores Primos}]
V(int) primeFactors(bigint number) {
    V(int) factors;
    bigint index = 0, pf = primos[index];
    while ((pf * pf) <= number) {
        while (n % pf == 0) {
            n /= pf;
            factors.PUB(pf);
        }
        pf = primos[++index];
    }
    if (number != 1) factors.PUB(number);
    return factors;
}
\end{lstlisting}

\newpage

\subsection{Matrizes}
\begin{lstlisting}[language=C++]
//#define MOD 1234567891
#define MOD 1000000007
M(int) matrixUnit(int n) {
    M(int) res(n, V(int)(n));
    for (int i = 0; i < n; i++) res[i][i] = 1;
    return res;
}

M(int) matrixAdd(const M(int) &a, const M(int) &b) {
    int n = a.size();
    int m = a[0].size();
    M(int) res(n, V(int)(m));
    for (int i = 0; i < n; i++)
        for (int j = 0; j < m; j++)
            res[i][j] = (a[i][j] + b[i][j]) % MOD;
    return res;
}

M(int) matrixMul(const M(int) &a, const M(int) &b) {
    int n = a.size();
    int m = a[0].size();
    int k = b[0].size();
    M(int) res(n, V(int)(k));
    for (int i = 0; i < n; i++)
        for (int j = 0; j < k; j++)
            for (int p = 0; p < m; p++)
                res[i][j] = (res[i][j] + (big)
                			(a[i][p] * b[p][j]) % MOD) % MOD;
    return res;
}

M(int) matrixPow(const M(int) &a, int p) {
    if (p == 0) return matrixUnit(a.size());
    if (p & 1) return matrixMul(a, matrixPow(a, p - 1));
    return matrixPow(matrixMul(a, a), p / 2);
}

M(int) matrixPowSum(const M(int) &a, int p) {
    int n = a.size();
    if (p == 0) return M(int)(n, V(int)(n));
    if (p % 2 == 0)
        return matrixMul(matrixPowSum(a, p / 2),
        				 matrixAdd(matrixUnit(n), 
                        		   matrixPow(a, p / 2)));
    return matrixAdd(a, matrixMul(matrixPowSum(a, p - 1), a));
}
\end{lstlisting}

