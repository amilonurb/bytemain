\begin{lstlisting}[language=C++, title=Checa se é bipartido]
bool isBipartite() {
    color[0] = 1;
    queue<int> fila;
    fila.push(0);
    while (!fila.empty()) {
        int temp = fila.front();
        fila.pop();
        for (int i = 0; i < (int) grafo.size(); i++) {
            if (grafo[temp][i] && color[i] == -1) {
                color[i] = 1 - color[temp];
                fila.push(i);
            } else if (grafo[temp][i] && 
                       color[i] == color[temp])
                return false;
        }
    }
    return true;
}
\end{lstlisting}

\newpage

\begin{lstlisting}[language=C++, title=Emparelhamento Máximo em Grafo Bipartido]
#define rep(i, n) for (int i = 0; i < n; i++)
#define MAX_N 100
m(int) grafo;
v(bool) visitados;
bool FindMatch(int i, v(int) &left, v(int) &right) {
    rep(j, grafo[i].size()) {
        if (grafo[i][j] && !visitados[j]) {
            visitados[j] = true;
            if (right[j] < 0 || 
                FindMatch(right[j], left, right)) {
                left[i] = j;
                right[j] = i;
                return true;
            }
        }
    }
    return false;
}

int BipartiteMatching(v(int) &left, v(int) &right) {
    left = v(int)(grafo.size(), -1);
    right = v(int)(grafo[0].size(), -1);
    int ct = 0;
    rep(i, grafo.size()) {
        visitados = v(bool)(grafo[0].size());
        if (FindMatch(i, left, right)) ct++;
    }
    return ct;
}
\end{lstlisting}